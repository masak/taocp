\documentclass{article}
\usepackage[utf8]{inputenc}
\usepackage[english]{babel}
\usepackage{amsmath}

\begin{document}

We are given the two modular identities

\begin{align*}
x \bmod 3 &= 2 \\
x \bmod 5 &= 3
\end{align*}
The first thing we would like to do is to put them on ``common moduli''.
Law C allows us to do so:

\begin{align*}
5 x \bmod 15 &= 10 \\
3 x \bmod 15 &= 9
\end{align*}
Subtracting one from the other (using Law A):

\begin{equation*}
2 x \bmod 15 = 1
\end{equation*}
We are almost there; we'd like to get the $x$ alone on the left side.
If we were dealing with rationals or reals, we would just divide both sides by $2$ and be done.
But this is modular arithmetic; there isn't a division operation as such.
There is, however, the modular inverse of $2$, and we can multiply both sides by it.
Since $2 * 8 \equiv 1 \pmod{15}$, $8$ is the modular inverse of $2$.
Therefore,

\begin{equation*}
x \bmod 15 = 8
\end{equation*}
Done.

But wait!
How did we find that $8$, exactly?
By all accounts, we just pulled it like a rabbit out of a hat.
If we wanted to find it the hard way, we could scan for it among all the 15 possible values --- but that's not so elegant.
The extended Euclid's algorithm allows us to find it in just a few steps with these inputs:

\begin{equation*}
a m + b n = 1
\end{equation*}
In our case, $m = 2$, $n = 15$, and Euclid's algorithm gives us $a$ (and $b$) as output.
The file \texttt{modular.hs} contains a program with a demonstration of this solution.
Its function \texttt{modularInverse} has been simplified to not even compute $b$, as we don't care about it.

\end{document}
