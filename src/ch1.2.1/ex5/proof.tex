\documentclass{article}
\usepackage[utf8]{inputenc}
\usepackage[english]{babel}

\usepackage{amsthm}

\newtheorem{theorem}{Theorem}

\begin{document}

\begin{theorem}
All integers $> 1$ can be written as a product of one or more prime numbers.
\end{theorem}

\begin{proof}

We proceed by induction on the integer $n$.

\textbf{Base case}:
2 is prime, and so it can be written as a product of one or more primes.

\textbf{Inductive case}:
By assumption, we've shown that $2 \ldots n - 1$ can all be written as products of one or more prime numbers.
Let's now examine $n$.

\begin{itemize}

\item
\textbf{Case I}: $n$ is prime.
In this case, we are done; $n$ can be written as a product of one or more prime numbers.

\item
\textbf{Case II}: $n$ is not prime.
By definition of ``prime'', this means that $n$ \textit{does} have a divisor other than itself and $1$.
Let's name this divisor $p$.
Then there's also a number $q$ such that $p q = n$.
$p$ and $q$ can each be written as a product of one or more prime numbers; juxtaposing these gives $n$ as a product of primes.

\end{itemize}

\end{proof}

\end{document}
